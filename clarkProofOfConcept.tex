\documentclass{article}
\usepackage[utf8]{inputenc}
\usepackage{setspace}
\onehalfspacing

\title{Proof of Concept Scoping Exercise}
\author{Matthew Clark}
\date{}
\begin{document}

\maketitle
\newpage

\section{Jobs}
My thesis is exploring the use of creative machine learning frameworks in creating a collaborative Artificial Intelligence tool in the scope of electronic music. More specifically, I want to look into micro and macro rhythmic generation based on Techno music, and implementing a system to maximise human-computer interactivity. The jobs that will be required of me to produce this thesis include:
\begin{enumerate}
    \item Research into the social, historical, and cultural context of techno music
    \item Research pre-existing studies in fields of musical rhythm and techno music
    \item Define a scope for a data set of techno songs
    \item Collect a data set of techno songs based on my scope
    \item Analyse rhythmical structures within each song, based on:
    \begin{itemize}
        \item Tempo
        \item Rhythm
        \item Texture
        \item Structure
        \begin{itemize}
            \item Micro structures/loops
            \item Macro structures in full songs
        \end{itemize}
        \item Timbre of samples used
        \item Production techniques used
    \end{itemize}
    \item Theorize underlying rules that techno music follows
    \item Consolidate findings into a format for a thesis presentation
    \item Consolidate findings into a format for machine learning (ie: convert audio files into MIDI tracks
    \item Research into various machine learning algorithms and frameworks
    \item Learn the basics of TensorFlow, Python, and/or Javascript
    \item Experiment with different frameworks \& the inner structures of \\each framework
    \item Run frameworks to produce various outputs at different stages
    \item Analyse outputs produced to the data set and rules previously described
    \item Identify the best framework to use within a live production setting
    \item Export framework and data into a standalone VST
    \item Transform VST into a Max4Live patch
    \item Import Max4Live patch into Ableton Live
    \item Theorize software-based musical interfaces to maximise creativity and collaboration between human input and AI input
    \item Create a software-based interface based on previous researched theories in human-computer interaction and new musical interfaces
    \item Analyse the production result
    \item Discuss the result
    \item Keep track of all songs used in my data set and research used for a bibliography and discography
    \item Writing up the thesis in a format that can take into account appendices which include musical notation, images of graphs, references to spreadsheets, snippets of code, screen-shots of programs, and references to audio files.
\end{enumerate}
\subsection{Project Description}
My thesis is a interdisciplinary approach, including musicology fields such as musical analysis, computer science fields such as machine learning and programming, and interdisciplinary sub-fields such as Machine Information Retrieval (MIR), Human-Computer Interaction (HCI), and New Musical Interfaces (NMI). The decision with using Electronic Dance Music (EDM) as my focus is due to its prevalence in modern society due to the exponential increase of sophistication and accessibility of musical production software and hardware, which also lends itself nicely to a genre of music that itself is interdisciplinary, combining computer science, mathematics, and musicology. The reason I have decided to go with Techno is its overall simplicity, yet contains many sub-genres which all contain minor variations on the traditional form of Techno, lending itself well to a MRes scale. I have narrowed the scope again to only rhythmical components the underlying percussion used, as it removes the complexity of tonality.\\
The main reason this project is a contribution to knowledge is the rapid development of Artificial Intelligence (AI) due to technological developments, and the recent deployment of Google's Magenta project means that this high level of AI development is becoming more accessible with a higher degree of success in creating creative, novel outputs. The interdisciplinary field of computer music has always been a cutting edge field of combining both musicology and computer science, and its lack of applications to the field of EDM performance means this project will help develop a new field of combining AI and EDM, which also entails a large philosophical debate surrounding AI, authorship, and creativity. \\
This project also reveals potential uses for technical solutions, including identifying, gathering, and analysing large musical data sets, alongside MIR analysis of audio files and converting them into MIDI files. 
\subsection{Scope}
The scope for my FOAR705 project will include points 3-8 found in the \textbf{Jobs} section, with any machine learning production occurring if time permits, however there is a good chance it will fall out of the scope for this project, including the creation of a musical interface within Ableton Live. \\
The desired outputs include:
\begin{itemize}
    \item A data set which containing a list of techno songs defined by my scope, which will be developed as per point 3 in \textbf{Jobs}, which will include Artist, Country of Origin, Song, Date Released, Sub-Genre, Label, Sales History, and site that the data was extracted from. This will be presented in an excel spreadsheet
    \item A collection of MIDI data which reflects the percussion structure throughout each track, with a link to each file in the previous spreadsheet
    \item A data set which contains each song's structure, laid out in bars, featuring all the musical structures used throughout the song, contained within a spreadsheet.
    \item A spreadsheet with formalized rules and their frequency of occurrence within the data set 
\end{itemize}
This scope may reduce in size based on time restraints and technical capabilities, however the aim is to work through the jobs one by one until the end of semester.
\section{Pains}
Pains that I can identify in the process of my research include:
\begin{itemize}
    \item Losing track of references \& bibliographic material during my research (this has happened many times so far in my MRes)
    \item Having to manually identify the scope of the data set (Where and how do I identify what artists/songs I should use, and how do I collect and synthesise all that data without having to manually type out every entry from every web page I can find)
    \item Having to manually buy/download every song for analysis
    \item Analysing the rhythmical context of each individual song
    \item Extracting valuable key similarities from my analysis to devise general stylistic rules that my data set follows without having to do it by eye
    \item Converting the analytics of each song into a MIDI format
    \item General pitfalls when implementing TensorFlow in Python \& getting the machine learning algorithm to correctly function
    \item Potential errors that will occur when transferring the framework between different coding environments
    \item Having to analyse output material against the source material \& devised rules manually
    \item Losing track of versions of my thesis, along with my appendices' source material
\end{itemize}
\subsection{Pain Relievers}
Pain relievers which will help with time management and scalability for the above potential pains include:
\begin{itemize}
    \item Use a bibliography-based program to help consolidate and keep track of all my resources used, and which will create a bibliography automatically with proper formatting for my paper.
    \item After doing my own manual research, being able to use a pre-existing script or creating my own script that will extract all the relevant data I need, while keeping track of all the songs \& their sources
    \item Using a system that will allow me to download all the songs gathered in my data set automatically with correct file names and categorize them efficiently based on the numbering given in my data set file
    \item Use pre-existing software or look into new Musical Information Retrieval (MIR) systems that will allow me to extract the necessary data from each song without having to do it manually/by hand
    \item Using a pre-existing script or creating my own script that will automatically analyze the musical analysis data for key similarities and document them based on frequency within the musical data set
    \item Use pre-existing software or devising a new script to convert the musical analysis into MIDI format
    \item Use a learning journal to document my experiences in TensorFlow and Python
    \item Use a learning journal to document my experiences in the machine learning phase
    \item Use a learning journal to document my experiences in the framework conversion process
    \item Using a pre-existing script or developing a new script to help compare the output of the AI program against the source material and devised rules. 
    \item Use a learning journal overall in my entire process to both document my methodology, noting useful key pieces of software used, scripts that were developed, and errors that I occurred to avoid repeating the same mistakes.
    \item Use Github to keep track of versions of my paper and appendices/data sets
\end{itemize}

\section{Gains}
Gains that would help improve my thesis include:
\begin{itemize}
    \item Being able to download bulk songs for analysis without having to pay full retail price for each individual song digitally, or being able to analyse them from a stream.
    \item Have a bibliography program which can use my department-specific style for both bibliography and discography
    \item I would rather be able to access pre-existing built scripts and software with minimal adjustments to extract data rather than having to create a script or program from scratch
    \item Using MIR software that can analyse music with a high \% of accuracy, so less quality control checks are needed.
    \item Setting up software and files in such a way that minimizes potential data loss and time spent navigating folders and files
    \item Being able to find quick and easy tutorials to get me up to an adequate speed in TensorFlow and Python
    \item Doing adequate prior research before attempting any sort of machine learning to maximise time to analyse outputs and minimise using a trial-and-error approach to inner layers and general frameworks
    \item An aspiration I want to achieve is creating a system that can be used as if I were producing next to another human instead of using the AI as a tool in the production/performance
    \item Success is measured by being able to create a system that can be easily scaleable (eg: go from using 100 songs to using 1000 songs; analysing rhythm to analysing harmony/melody), and being able to create a framework that produces usable results
    \item Increasing the value proposition of my thesis results in being able to automate analytical tasks to free up more time for non-automatable tasks (eg: writing out my paper, learning different programming languages), and being able to count on a high degree of accuracy from these automatable tasks.
\end{itemize}
\subsection {Gain Creators}
Ways of creating these gains as stated above include:
\begin{itemize}
    \item Investigate the purchasing of music in bulk under educational licenses or through a cheaper service rather than iTunes/Amazon/Google etc.
    \item Look into and start implementing a bibliography program that allows for custom editing of style \& also accounts for a discography
    \item Do adequate research into the various programs and scripts available for MIR \& data gathering before starting a new section of my thesis
    \item Start using Github to backup projects, arrange files, folders, and programs alongside productivity programs such as Trello/Todoist to maximise productivity
    \item Start learning TensorFlow and Python prior to the machine-learning phase
    \item Do adequate research into machine learning frameworks from academic journals such as the Computer Music Journal, and Google's Magenta project
    \item Objectively striving for creating tasks that can be automated to improve scalability for future research outside of an MRes 
\end{itemize}

\end{document}
